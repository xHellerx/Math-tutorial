\chapter*{Введение}
\addcontentsline{toc}{chapter}{Введение}

\section*{Что это?}
Это курс математики без какой-либо конкретной тематики. Просто сборник глав, которые я считаю небесполезными для ознакомления.

\section*{Зачем это?}
Целей я преследую две.

Во-первых, моя личная философия утверждает, что смысл жизни заключается в получении удовольствий. Я так же совершенно точно знаю, что занятия математикой могут приносить удовольствие мало с чем сравнимое, хотя прежде чем этот момент настанет, пройдет какой-то длительный и сложный период времени. Моя цель таким образом --- познакомить читателя с математикой в тех ее аспектах, которые на мой взгляд раскрывают красоту науки.

Во-вторых, я убежден, что занятия математикой благотворно влияют на умственное здоровье, и соответственно что эта книга может оказаться полезной читателю так же как ему мог бы быть полезен спорт зал.

\section*{Тут же всё не понятно!}
Разбирайтесь. Две первых проблемы при изучении математики --- это отсутствие ответа на вопрос «Зачем это надо», а так же отсутствие развитого абстрактного мышления. И то и другое приходит со временем.

Для развития абстрактного мышления нужно решать задачи и пытаться воспроизводить доказательства. На вопрос «Зачем это надо» ответ можно получить лишь уже имея какую-то начальную базу. После какого-то периода времени всё это придет.

Если вам не понятен какой-то абзац текста --- попробуйте решать упражнения в конце параграфа, они обычно дублируют материал параграфа, но подразумевают более подробный самостоятельный анализ. Если и после этого вам ничего не понятно --- пытайтесь читать снова. Если опять не понимаете --- пропускайте и идите дальше. Если не понятно совсем никак ни в какую --- спросите у кого-нибудь. Если ничего не помогает --- пишите мне, скорее всего я сам допустил ошибку, не математическую так методическую.

Имейте так же ввиду, что каждая глава начинается простыми фактами, и постепенно скатывается на глубокие теоретические тонкости, которые возможно многим покажутся необязательными при первом знакомстве с материалом. По этой причине непонимание отдельных параграфов на страшно. К ним всегда можно вернуться впоследствии. Это не особая характеристика этой книги, а общий принцип чтения большинства математических текстов.

\section*{И что я должен все запоминать?}
Ни в коем случае. В этой книге сотни определений и теорем. Заучить их невозможно, да и не нужно. Надо пытаться понять на интуитивном уровне излагаемый материал и усваивать ход мысли. Если что-то не понятно и не удается понять --- смело пропускайте. Позже можно вернуться, когда будет понятно где это и как используется. Задача при прочтении книги не запомнить всё и сдать по ней экзамен, а уловить общую суть.

\section*{На кого ориентирован курс?}
Курс ориентирован на всех. Начальных требований для изучения курса никаких нет --- математика тут излагается с нуля, причем не с арифметики, как можно было бы подумать, а с философских ее оснований. Желательно лишь знать базовую арифметикику в виде сложения, умножения, вычитания и деления --- остальное будет раскрыто в книге.

\section*{Поможет ли книга сдать экзамены?}
Нет. Курс не соответствует ни школьной программе, ни институтской, хотя некоторые пересечения и имеются, но в минимальных дозах. Основная разница между программой ВУЗов и школ и эти курсом заключается в том, что в школе и институте учат решать конкретые задачи, а я себе ставлю целью научить читателя самостоятельно придумывать решения к задачам самым различным. Таким образом именно для подготовки к экзамену курс явно не подойдет, но его заблаговременное изучение поможет лучше понимать что происходит на доске.

\section*{А почему тут так мало?}
Учебник я пишу по параграфам и выкладываю тут же в свой блог http://heller.ru/blog/. После комментариев читаталей, перерабатываю материал и компоную это все главы в формате pdf. Учебник постоянно расширяется, в старых материалах правятся ошибки и они пополняются упражнениями. Непонятные по отзывам читателей места перерабатываются или вообще переписываются заново.

\section*{Могу ли я помочь проекту?}
Да. Во-первых, всегда требуется помощь с поиском ошибок в тексте (опыт показывает, что их полно) и с общей критикой. Во-вторых, требуется помощь с версткой. Если вы хорошо разбираетесь в \LaTeX, я был бы признателен за помощь с оформлением, так как у меня у самоо с этим очень плохо. В-третьих, было бы полезно рспространение информации о курсе. Посоветуйте его в своем блоге, своим студентам, родителям и соседям.
