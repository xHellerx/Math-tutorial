\chapter*{Введение}
\addcontentsline{toc}{chapter}{Введение}

Эта книга изначально замышлялась как простой учебник для новичков в математике. После первых двух глав (которые планировалось сделать коротким введением и обзором, но в итоге они разрослись на сотню страниц) стало понятно, что этот учебник понимают только единицы. Увы. Тем не менее учебник я продолжаю писать и буду по мере возможностей его перерабатывать, чтобы сделать более доступным.

Пока можно порекомендовать обращаться к отдельным главам учебника и читать материал поверхностно. Если что-то понятно~--- замечательно. Если непонятно~--- не беда. Либо это вам и не нужно, либо прочитаете то же самое в другом месте на более простом, менее научном языке. Несмотря на сложность, я считаю, что в принципе чтение его может оказаться полезным, в том числе и самым новичкам.

Учебник пишется на чистом энтузиазме, распространяется свободно и без ограничений.

Для скачивания всегда доступна последняя версия pdf-файла здесь: \url{http://heller.ru/tutorial.pdf}

Исходные коды в \LaTeX{} могут быть скачаны на GitHub: \linebreak
\url{http://github.com/xHellerx/Math-tutorial}

Также учебнику всегда требуется помощь с поиском ошибок в тексте (опыт показывает, что их полно) и с общей критикой. Также требуется помощь с вёрсткой. Если вы хорошо разбираетесь в \LaTeX, я был бы признателен за помощь с оформлением, так как у меня у самого с этим очень плохо. В-третьих, было бы полезно распространение информации о курсе. Посоветуйте его в своём блоге, своим студентам, родителям и соседям.
