\section{Другие логики}

Та логика, которую мы рассматривали до сих пор, сама по себе на самом деле не имеет никакой монополии на то, чтобы быть единственно истинной. Она удобна и правдоподобна почти во всех ветвях математики, однако помимо неё существует множество других разновидностей логики. В этом параграфе мы ознакомимся с некоторыми из них очень кратко и главным образом неформально, исключительно для того, чтобы у читателя сложилось какое-то впечатление. В дальнейшем эта логика нам в курсе не понадибтся кроме темпоральной и эпистемиальной, но даже без этого параграфа дальнейший материал будет понятен.

Логика, которую мы рассматривали до сих пор называется \term{классической логикой}, что означает, что любое выражение в ней обязательно либо истинно либо ложно. Внутри самой классической логики так же есть градация: если не рассматривать кванторов, то эта логика называется \term{логикой высказываний}, а вместе с кванторами она называется \term{логикой первого порядка}. Эта логика допускает выражения вида $\forall P(x)$, но не допускает выражений $\forall P, P(x)$. Если допустить последнее (то есть разрешить не только выражения типа <<для любого объекта $x$>>, но и выражения <<для любого предиката $P$>>), то такая логика будет называться \term{логикой второго порядка}. Но всё это разновидности классической логики. В математике практически всегда дело ограничивается классической логикой первого порядка.

В этой главе у нас пойдёт речь о неклассической логике. Простейший пример, когда возникает нужда в такой логике~--- это компьютерные базы данных. Базу данных можно представить себе как набор таблиц с какой-то информацией. Для определенности будем считать, что мы имеем таблицу участников накопительной программы в косметическом салоне. Среди прочих данных в таблице участников имеется графа <<возраст>>, который участники программы могут сообщать, а могут и нет. То есть эта графа может быть пустой. Это вполне реальная ситуация и любая база данных обычно имеет специально выделенное значение NULL, которым забиваются те данные, которыми мы не располагаем или которые вообще неопределены.

Пусть предикат $Y(a, b)$ означает, что участник акции $a$ моложе участника акции $b$. Этот предикат не вызывает вопросов до тех пор, пока мы сравниваем участников, которые сообщили возраст. А что должна вывести программа, если мы задали ей вычислить этот предикат для участников, которые свой возраст не сообщили? Значение этого предиката не определено и мы приходим к необходимости помимо истины (1) и лжи (0) ввести так же понятие неопределенности ($U$) в нашу логику.

Когда мы ввели новое логическое значение, мы должны определить как с этим значением будут работать логические операции. Сделать это возможно многими способами, самый простой и естественный из которых называется \term{логикой Клини} и именно она чаще всего реализована в базах данных. Чаще всего в учебниках для программистов на неё ссылаются просто как на \term{третичную логику}, но это не совсем корректно: третичной логикой называется любая логика, в которой есть три значения истинности. Значения истинности приведены в таблицах 1.10, 1.11, 1.12 и 1.13.

\begin{table}[h]
\centering
\begin{tabular}{c | c}
$a$ & $\neg b$ \\
\hline
0 & 1 \\
U & U\\
1 & 0
\end{tabular}
\caption{Связка <<НЕ>> в логике Клини}\label{table:kleene-not}
\end{table}

\begin{table}[h]
\centering
\begin{tabular}{c | c c c}
$\land$ & 0 &U &1 \\
\hline
0 & 0 & 0 & 0 \\
U & 0 & U & U\\
1 & 0 & U & 1
\end{tabular}
\caption{Связка <<И>> в логике Клини}\label{table:kleene-and}
\end{table}

\begin{table}[h]
\centering
\begin{tabular}{c | c c c}
$\lor$ & 0 &U &1 \\
\hline
0 & 0 & U & 1 \\
U & U & U & 1\\
1 & 1 & 1 & 1
\end{tabular}
\caption{Связка <<ИЛИ>> в логике Клини}\label{table:kleene-or}
\end{table}

\begin{table}[h]
\centering
\begin{tabular}{c | c c c}
$\to$ & 0& U& 1 \\
\hline
0 & 1 & 1 & 1 \\
U & U & U & 1\\
1 & 0 & U & 1
\end{tabular}
\caption{Импликация в логике Клини}\label{table:kleene-or}
\end{table}

Проработайте эти таблицы и попытайтесь понять почему они именно такие, а не какие-то другие. Однако надо иметь ввиду, что это не единственный вариант третичной логики. Самый распространенный альтернативный вариант~--- это \term{логика Лукаcевича}, которая отличается от логики Клини лишь тождеством $U\to U = 1$. Проблема логики Клини в том, что никое предложение в нём не может быть всегда истинным. Например, в классической логике мы имели полезнейший закон де Моргана
$$\neg(a \land b) \leftrightarrow \neg a \lor \neg b$$
а в логике Клини он уже не работает, если вспомнить, что эквивалентность задаётся как
$$(a \leftrightarrow b) = (a\to b)\land (b\to a)$$
Более того: в логике Клини нет вообще ни одной тавтологии. Логика же Лукасевича хоть и не сохраняет все законы классической логики (это было бы и невозможно), по крайней мере вводит хоть какие-то теоремы.

\begin{exercise}
Докажите в логике Лукасевича, что
$$(a\lor b) \leftrightarrow (a \to b) \to b$$
\end{exercise}

\begin{exercise}
Докажите в логике Лукасечива закон де Моргана.
\end{exercise}

\begin{exercise}
Докажите в логике Лукасевича закон двойного отрицания
$$\neg\neg a = a$$
\end{exercise}

\begin{exercise}
Докажите, что в логике Лукасевича не работает закон исключенного третьего
$$a\lor \neg a = 1$$
\end{exercise}

\begin{exercise}
Докажите, что в логике Лукасевича не работает закон противоречия
$$a\land \neg a = 0$$
\end{exercise}

\begin{exercise}
Докажите, что в логике Клини нет ни одной тавтологии, использующей только переменные и приведенные логические операции (если мы будем вводить новые операции, то мы очевидно можем подогнать тавтологии под бесполезные операции~--- это не интересно совершенно).
\end{exercise}

