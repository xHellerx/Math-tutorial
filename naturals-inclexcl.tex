\section{Включения-исключения}

Вот задачка, которая как-то попалась мне в младшем возрасте на олимпиаде по математике, а затем многократно попадалась в разных книжках с занимательными примерами по математике.

Есть класс, в котором известно, что все ученики ходят в какие-то секции. Известно, что 12 учеников ходят на плавание (и, возможно на что-то ещё), 13 на карате, 12 на шахматы, 6 одновременно на плавание и на карате, 6 на карате и шахматы, 2 на шахматы и плавание и 2 ходят во все три секции. Сколько всего учится детей в классе?

Мы не будем решать эту задачу, а сразу перейдём к общему случаю. Пусть у нас есть семейство множеств $\{A_i\}$, нам известны мощности каждого из них, а такжТакже мощности любых их пересечений. Необходимо найти мощность объединения этих множеств.

Напомню, что в~\S~3.1 мы ввели понятие характеристической функции $\chi_A(x)$. Эта функция принимает значение 0, если $x\not\in A$ и 1, если $x\in A$. Нам потребуются дополнительные факты о ней:

\begin{thm}
\begin{enumerate}
\item Пусть $A\subset U$, тогда $|A| = \sum_{x\in U} \chi_A(x)$
\item $\chi_{A^C}(x) = 1 - \chi_A(x)$
\item $\chi_{A\cap B}(x) = \chi_A(x)\chi_B(x)$ 
\end{enumerate}
\end{thm}
\begin{proof}
В качестве упражнения.
\end{proof}

\begin{thm}
$$\left|\bigcup_{i=1}^n A_i \right|= \sum_{i=1}^n|A_i| - \sum_{i<j}|A_i\cap A_j| + \sum_{i<j<k}|A_i\cap A_j\cap A_k| -\ldots$$
Здесь суммирование справа ведётся по всем таким наборам $(i, j)$ таким, что $i<j$, затем по наборам $(i,j,k)$, таким что $i<j<k$ и так далее.
\end{thm}

\begin{proof}
Утверждение на самом деле следует непосредственно из теоремы~3.32. В выкладках ниже я не указываю пределы суммирования, т.к. они и так очевидны:
\begin{align*}
\chi_{\cup A_i}(x) &= \chi_{(\cup A_i)^{CC}}(x) \\
&= 1 -  \chi_{(\cup A_i)^C}(x) \\
&= 1 - \chi_{\cap A_i^C}(x) \\
&= 1 - \prod \chi_{\cap A_i^C}(x) \\
&= 1 - \prod (1 - \chi_{A_i}(x)) \\
&= \sum_{i=1}^n\chi_{A_i}(x) - \sum_{i<j}\chi_{A_i}(x)\chi_{A_j}(x) +\ldots\\
&= \sum_{i=1}^n\chi_{A_i}(x) - \sum_{i<j}\chi_{A_i\cap A_j}(x) +\ldots
\end{align*}
В предпоследней строке мы просто раскрыли скобки. Применяя к полученному пункт 1 теоремы 3.32, получаем утверждение теоремы.
\end{proof}

Теорема легко выводится, но пока, вероятно, не очень понятна по смыслу. Давайте посмотрим на случаи нескольких множеств. Пусть нам надо подсчитать $|A\cup B|$. Мы могли бы просто сложить их мощности $|A| + |B|$, однако в этом случае получится, что часть $A\cap B$ будет учтена в этом выражении дважды~--- один раз от $|A|$ и  один раз от $|B|$, значит из полученного нам надо вычесть $|A\cap B|$, в результате чего получаем

\begin{equation}\label{nii:1}
|A\cup B| = |A| + |B| - |A\cap B|
\end{equation}

Можно было бы увидеть это и по-другому. Множества $X = A\backslash B$, $Y = A\cap B$ и $Z = B\backslash A$ не пересекаются, причём $A = X\cup Y$, $B = Y\cup Z$, $A\cup B = X\cup Y \cup Z$. Т.к. $X$, $Y$ и $Z$ не пересекаются, мы можем записать
$$|A\cup B| = |X\cup Y\cup Z| = |X| + |Y| + |Z|$$
Но поскольку $|A| = |X| + |Y|$, $|B| = |Y| + |Z|$ и $|A\cap B| = |Y|$, мы опять получаем~\eqref{nii:1}.

Перейдём к случаю трёх множеств. Опять же простая сумма $|A|+|B|+|C|$ будет дважды учитывать попарные пересечения множеств и трижды~--- тройное пересечение $A\cap B\cap C$. Чтобы исправить ситуацию, нам надо вычесть все попарные пересечения, однако этого тоже недостаточно~--- вычтя три попарных пересечения мы в том числе три раза вычтем тройное пересечение, а это значит, что оно теперь остаётся вообще не учтено. Его надо прибавить. Получаем

$$|A\cup B \cup C| = |A| + |B| + |C| - |A\cap B| - |A\cap C| - |B\cap C| + |A\cap B \cap C|$$

Собственно утверждение теоремы 3.33 можно было бы получить и по индукции, рассуждая таким образом, однако такой путь явно сложнее.

Для пересечения множеств можно получить результат, аналогичный 3.33.

\begin{thm}
Пусть $A_i \subset U$, тогда
$$\left|\bigcap_{i=1}^nA_i\right| = |U| - \sum_{i=1}^n|A_i^C| + \sum_{i<j}|A_i^C\cap A_j^C| - \ldots$$
\end{thm}
\begin{proof}
Утверждение следует непосредственно из тождества
$$|\cap A_i| = |(\cap A_i)^{CC}| = |U| - |\cup A_i^C|$$
Применяя теперь теорему 3.33 получаем желаемый результат.
\end{proof}

Помимо простых задачек для школьников, конечно же, это всё имеет применения и к чистой математике. Давайте приведём примеры.

\begin{definition}
\term{Функцией Эйлера} $\phi(n)$ называется количество чисел, меньших $n$ и взаимопростых с ним (исключая ноль, включая единицу).
\end{definition}

\begin{thm}
Пусть $n=\prod_{i=1}^m p_i^{\alpha_i}$~--- разложение $n$ на простые множители. Тогда
$$\phi(n) = n - \sum_{i=1}^m\frac{n}{p_i} + \sum_{i<j}\frac{n}{p_ip_j} - \sum_{i<j<k}\frac{n}{p_ip_jp_k} + \ldots$$
\end{thm}
\begin{proof}
Будем рассматривать в качестве универсума (то есть множества, относительно которого мы берём дополнения) множество чисел от 1 до $n-1$. Определим $B_i$ как множество чисел, делящихся на $p_i$ и меньших $n$. Его дополнение $B_i^C$~--- это в точности числа, меньшие $n$, которые не делятся на $p_i$. Соответственно числа, взаимопростые с $n$~--- это пересечение $\cap B_i^C$, то есть это все числа, которые не делятся ни на один из делителей числа $n$. Заметим, что $|B_i| = \frac{n}{p_i}$.  в силу того, что все $\{p_i\}$ взаимопросты, $|B_i\cap B_j| = \frac{n}{p_ip_j}$ (в это пересечение попадают только числа, кратные $p_ip_j$). Аналогичные утверждения можно получить для любых пересечений множеств $\{B_i\}$. Отсюда сразу же по теореме 3.34 вытекает требуемое утверждение.
\end{proof}

\begin{exercise}
\term{Беспорядком} называется такая перестановка, которая не оставляет на месте ни один элемент множества, то есть такая, что $\rho(x)\not=x$. Сколько беспорядков возможно задать на множестве $[n]$?
\end{exercise}

\begin{thm}
Существует в точности
$$k^n - {k \choose 1}(k-1)^n + {k\choose 2}(k-2)^n -\ldots$$
сюръекций вида $[n]\to[k]$ (напомню, что это функции $f$ такие, что для каждого $y\in[k]$ существует $x$ такой что $f(x) = y$).
\end{thm}
\begin{proof}
Пусть множество $B_i$~--- это множество таких отображений $f:[n]\to[k]$, что прообраз элемента $i$ не пуст (то есть существует такое $x$, что $f(x) = i$). Множество сюръекций~--- это в точности множество $\cap B_i$. Чтобы применить теорему~3.34 нам необходимо найти мощность всех возможных пересечений множеств семейства $\{B_i^C\}$. Однако, это легко: если $I\subset [n]$, то
$$\left|\bigcap_{i\in I} B_i^C\right| = (k-|I|)^n$$
поскольку это в точности множество отображений $[n]\to([k]\backslash I)$. Причём это выражение зависит только от количества множеств в пересечении $|I|$, а не от самих множеств, которые пересекаются. Отсюда получаем, что для каждого $m = |I|$ мы будем иметь $k\choose m$ слагаемых $(k-m)^n$.
\end{proof}

\begin{exercise}
Предположим, что компьютер запрограммирован выполнить $n_i$ задач типа $i$, всего есть $m$ типов задач. Компьютер выполняет задачи последовательно, но ему надо так распланировать свою работу, чтобы все задачи одного вида не выполнялись последовательно, он обязательно должен задачи чередовать (то есть он не может выполнить подряд $n_i$ задач типа $i$). Сколько всего способов существует для компьютера распланировать свою работу?
\end{exercise}
