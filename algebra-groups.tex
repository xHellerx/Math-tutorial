\section{Группы}

\begin{definition}
	Множество с заданной на нём ассоциативной операцией называется \term{полугруппой}.
\end{definition}

\begin{definition}
	Если полугруппа имеет нейтральный элемент, то она называется \term{моноидом}.
\end{definition}

\begin{definition}
	\term{Группой} называется полугруппа, в которой есть нейтральный элемент, и каждый элемент имеет обратный.
\end{definition}

\begin{definition}
	Если бинарная операция группы коммутативна, то группа называется \term{абелевой} (реже просто \term{коммутативной}).
\end{definition}

\begin{definition}
	Подмножество группы, само являющееся группой относительно той же бинарной операцией, называется \term{подгруппой}.
\end{definition}

Аналогично можно определить подмоноиды, подполугруппы, дальше у нас появятся кольца и соответственно подкольца, поля и подполя, модули и подмодули и так далее. Мы не будем каждый раз вводить отдельное определение, считая значение приставки <<под>> очевидным.

Из всего сказанного нас будут главным образом интересовать группы. Абелевы группы кажутся очень удобными, но на практике с ними мы сталкиваться будем куда реже, чем с группами, операция которых некоммутативна.

Бинарную операцию на группе будем пока записывать в мультипликативной форме, поскольку большинство примеров этого параграфа обычно используются в литературе именно такую форму записи.

\begin{example}
	\term{Тривиальной} группой называется группа, состоящая из единственного элемента, являющегося нейтральным.
\end{example}

\begin{example}
	Множество $S_n$ перестановок $n$-элементного множества, образует группу относительно композиции с некоммутативной операцией. Более общо множество биекций вида $X\to X$ на произвольном множестве $X$ (не обязательно конечном) образуют группу.
\end{example}

\begin{thm}
	В любой группе $(ab)^{-1}=b^{-1}a^{-1}$.
\end{thm}
\begin{proof}
	\[
	(ab)(b^{-1}a^{-1}) = a(bb^{-1})a^{-1} = aa^{-1} = 1
	\]
\end{proof}

Обратите внимание, что формулировка теоремы, как и её доказательство, полностью дублируют теорему~\ref{thm:perminv} для перестановок. Таким образом мы ту теорему смогли обобщить на случай произвольных групп. Ещё один плюс абстракции, хоть и опять довольно слабенький.

\begin{exercise}
	Докажите, что множество обратимых элементов моноида образует группу.
\end{exercise}

\begin{exercise}\label{ex:grpint}
Даны группы $G$ и $H$, пересекающиеся как множества. Более того, на пересечении $G\cap H$ их операции совпадают. Докажите, что $G\cap H$~--- группа. Докажите, что это же верно для произвольного семейства групп, в том числе бесконечного.
\end{exercise}

\begin{example}
	Пусть $S$~--- произвольное множество (будем называть его \term{алфавитом}). Введём множество $S^{-1}$ <<обратных>> элементов к $S$. Слово <<обратный>> в данном случае неформальное, просто если $x\in S$, мы во множествe $S^{-1}$ будет элемент $x^{-1}$. На $S$ может не быть определено никаких бинарных операций, так что в данном случае это не более чем обозначение.
	
	Любую конечную последовательность символов из $S\cup S^{-1}$ будем называть \term{словом}. Будем считать, что стоящие рядом символы из $S$ и соответствующий ему символ из $S^{-1}$ можно <<сокращать>>: $abb^{-1}c = ac$. <<Произведением слов>> будем называть их обычную конкатенацию:
	\[
	abbcdc^{-1} \cdot cd^{-1}c = abbcdc^{-1}cd^{-1}c = abbcdd^{-1}c = abbcc
	\]
	
	Такое произведение задаёт группу на множестве слов. Этак группа называется \term{свободной} и обозначается как $F_S$. Докажите самостоятельно, что это группа.
\end{example}

Для удобства так же несколько идущих подряд символов можно записывать, указав просто число повторений сверху:
\[
aaabccdde^{-1}e^{-1}f = a^3 b c^2 d^2 e^{-2} f
\]

Пусть задана группа $F_S$ и помимо того, что $xx^{-1}=1$ какие-то ещё слова над этим алфавитом так же сокращаются. Полученное множество слов так же определяет группу (докажите это). Если $a, b, c$~--- буквы $S$, а $w, t$~--- слова этого алфавита, то группа, полученная путём <<сокращения>> этих слов в $F_S$ обозначается как $<a, b, c | w = t = 1>$.

\begin{example}
	Группа $<z|z^n = 1>$ называется \term{циклической группой порядка $n$} и обозначается как $Z_n^+$.
\end{example}

\begin{example}
	Рассмотрим $Z_{30}^+$. В этой группе $z^{15}z^{20} = z^5$, например.
\end{example}

\begin{thm}
	Пусть $A\subset G$~--- произвольное подмножество группы $G$. Тогда однозначно определена подгруппа $G$, наименьшая по размеру, содержащая $A$ как подмножество.
\end{thm}
\begin{proof}
	Рассмотрим семейство всех подгрупп $G$, содержащих $A$. По результату упражнения \ref{ex:grpint} пересечение всех групп этого семейства будет снова группой. Очевидно, что это наименьшая подгруппа $G$, содержащая $A$.
\end{proof}

\begin{thm}
	Пусть $G$~--- произвольная группа, и $x$~--- её элемент. Тогда подгруппа $<x>$ будет либо свободной группой одного элемента, либо циклической (таким образом произвольная конечная группа всегда содержит в себе циклическую подгруппу).
\end{thm}