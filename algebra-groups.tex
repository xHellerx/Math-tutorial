\section{Группы}

\begin{definition}
	Множество с заданной на нём ассоциативной операцией называется \term{полугруппой}.
\end{definition}

\begin{definition}
	Если полугруппа имеет нейтральный элемент, то она называется \term{моноидом}.
\end{definition}

\begin{definition}
	\term{Группой} называется полугруппа, в которой есть нейтральный элемент, и каждый элемент имеет обратный.
\end{definition}

\begin{definition}
	Если бинарная операция группы коммутативна, то группа называется \term{абелевой} (реже просто \term{коммутативной}).
\end{definition}

\begin{definition}
	Подмножество группы, само являющееся группой относительно той же бинарной операцией, называется \term{подгруппой}.
\end{definition}

Аналогично можно определить подмоноиды, подполугруппы, дальше у нас появятся кольца и соответственно подкольца, поля и подполя, модули и подмодули и так далее. Мы не будем каждый раз вводить отдельное определение, считая значение приставки <<под>> очевидным.

Из всего сказанного нас будут главным образом интересовать группы. Абелевы группы кажутся очень удобными, но на практике с ними мы сталкиваться будем куда реже, чем с группами, операция которых некоммутативна.

Бинарную операцию на группе будем пока записывать в мультипликативной форме, поскольку большинство примеров этого параграфа обычно используются в литературе именно такую форму записи.

\begin{example}
	\term{Тривиальной} группой называется группа, состоящая из единственного элемента, являющегося нейтральным.
\end{example}

\begin{example}
	Множество $S_n$ перестановок $n$-элементного множества, образует группу относительно композиции с некоммутативной операцией. Более общо множество биекций вида $X\to X$ на произвольном множестве $X$ (не обязательно конечном) образуют группу.
\end{example}

\begin{example}
	В таблице \ref{tb:v4} приведена четвёртая группа Клейна $V_4$.
\end{example}

\begin{table}[h]
	\centering
	\begin{tabular}{c|cccc}
		& $1$ & $a$ & $b$ & $c$ \\ 
		\hline $1$ & $1$ & $a$ & $b$ & $c$ \\ 
		$a$ & $a$ & $1$ & $c$ & $b$ \\ 
		$b$ & $b$ & $c$ & $1$ & $a$ \\ 
		$c$ & $c$ & $b$ & $a$ & $1$ \\ 
	\end{tabular}
	\caption{Четвёртая группа Клейна $V_4$}\label{tb:v4}
\end{table}

\begin{exercise}
	Докажите, что $V_4$ действительно задаёт группу.
\end{exercise}

\begin{example}
	Таблицы \ref{tb:cmplxgrp} и \ref{tb:qtgrp}, если я нигде не ошибся при заполнении, определяют группы на множествах $\{1, -1, i, -i\}$ и $\{1, -1, i, -i, j, -j, k, -k\}$ соответственно. Запоминать эти таблицы совершенно не нужно, эти группы мы рассмотрим далее отдельно в несколько другом контексте. Чтобы увидеть, что таблицы задают действительно группы, достаточно заметить соотношение $i^2 = j^2 = k^2 = ijk = -1$ и вывести отсюда все тождества, определяемые таблицей. Группа комплексных единиц является подгруппой группы кватернионных единиц. Интересно, что при этом умножение комплексных единиц коммутативно, а кватернионных единиц - нет.
\end{example}

\begin{table}[h]
	\centering
	\begin{tabular}{c|cccc}
		& $1$ & $-1$ & $i$ & $-i$ \\ 
		\hline $1$ & $1$ & $-1$ & $i$ & $-i$ \\ 
		$-1$ & $-1$ & $1$ & $-i$ & $i$ \\ 
		$i$ & $i$ & $-i$ & $-1$ & $1$ \\ 
		$-i$ & $-i$ & $i$ & $1$ & $-1$ \\ 
	\end{tabular} 
	\caption{Группа комплексных единиц}\label{tb:cmplxgrp}
\end{table}

\begin{table}[h]
	\centering
	\begin{tabular}{c|cccccccc}
		& $1$ & $-1$ & $i$ & $-i$ & $j$ & $-j$ & $k$ & $-k$ \\ 
		\hline $1$ & $1$ & $-1$ & $i$ & $-i$ & $j$ & $-j$ & $k$ & $-k$  \\ 
		$-1$ & $-1$ & $1$&$-i$ &$i$ & $-j$ & $j$ & $-k$ & $k$ \\ 
		$i$ & $i$ & $-i$ & $-1$ & $1$ & $k$& $-k$ & $-j$ & $j$ \\ 
		$-i$ & $-i$ & $i$ & $1$ & $-1$ & $-k$& $k$ & $j$ & $-j$ \\ 
		$j$ & $j$ & $-j$ & $-k$ &$k$ & $-1$& $1$ & $i$ & $-i$ \\ 
		$-j$ & $-j$ & $j$ & $k$ &$-k$ & $1$& $-1$ & $-i$ & $i$ \\ 
		$k$ & $k$ & $-k$ & $j$ & $-j$ &$-i$ & $i$  & $1$ &$-1$  \\ 
		$-k$ & $-k$ & $k$ & $-j$ & $j$ &$i$ & $-i$  & $-1$ &$1$  \\ 
	\end{tabular}
	\caption{Группа кватернионных единиц}\label{tb:qtgrp}
\end{table}

\begin{definition}
	Пусть $G$ и $H$~--- группы. \term{Прямым произведением групп} называется их прямое произведение $G\times H$ как множеств с заданной бинарной операцией
	\[
	(a, b) \cdot (c, d) = (ac, bd)
	\]
\end{definition}

\begin{exercise}
	Докажите, что прямое произведение групп является группой.
\end{exercise}

\begin{thm}
	В любой группе $(ab)^{-1}=b^{-1}a^{-1}$.
\end{thm}
\begin{proof}
	\[
	(ab)(b^{-1}a^{-1}) = a(bb^{-1})a^{-1} = aa^{-1} = 1
	\]
\end{proof}

Обратите внимание, что формулировка теоремы, как и её доказательство, полностью дублируют теорему~\ref{thm:perminv} для перестановок. Таким образом мы ту теорему смогли обобщить на случай произвольных групп. Ещё один плюс абстракции, хоть и опять довольно слабенький.

\begin{exercise}
	Докажите, что множество обратимых элементов моноида образует группу.
\end{exercise}

\begin{exercise}\label{ex:grpint}
	Даны группы $G$ и $H$, пересекающиеся как множества. Более того, на пересечении $G\cap H$ их операции совпадают. Докажите, что $G\cap H$~--- группа. Докажите, что это же верно для произвольного семейства групп, в том числе бесконечного.
\end{exercise}

\begin{example}
	Пусть $S$~--- произвольное множество (будем называть его \term{алфавитом}). Введём множество $S^{-1}$ <<обратных>> элементов к $S$. Слово <<обратный>> в данном случае неформальное, просто если $x\in S$, мы во множествe $S^{-1}$ будет элемент $x^{-1}$. На $S$ может не быть определено никаких бинарных операций, так что в данном случае это не более чем обозначение.
	
	Любую конечную последовательность символов из $S\cup S^{-1}$ будем называть \term{словом}. Будем считать, что стоящие рядом символы из $S$ и соответствующий ему символ из $S^{-1}$ можно <<сокращать>>: $abb^{-1}c = ac$. <<Произведением слов>> будем называть их обычную конкатенацию:
	\[
	abbcdc^{-1} \cdot cd^{-1}c = abbcdc^{-1}cd^{-1}c = abbcdd^{-1}c = abbcc
	\]
	
	Такое произведение задаёт группу на множестве слов. Этак группа называется \term{свободной} и обозначается как $F_S$. Докажите самостоятельно, что это группа.
\end{example}

Для удобства так же несколько идущих подряд символов можно записывать, указав просто число повторений сверху:
\[
aaabccddh^{-1}h^{-1}f = a^3 b c^2 d^2 h^{-2} f
\]

Пусть задана группа $F_S$ и помимо того, что $xx^{-1}=1$ какие-то ещё слова над этим алфавитом так же сокращаются. Полученное множество слов так же определяет группу (докажите это). Если $a, b, c$~--- буквы $S$, а $w, t$~--- слова этого алфавита, то группа, полученная путём <<сокращения>> этих слов в $F_S$ обозначается как $<a, b, c | w = t = 1>$.

\begin{example}
	Группа $<z|z^n = 1>$ называется \term{циклической группой порядка $n$} и обозначается как $Z_n^+$.
\end{example}

\begin{example}
	Рассмотрим $Z_{30}^+$. В этой группе $z^{15}z^{20} = z^5$, например.
\end{example}

\begin{example}
	$V_4 = <a, b | a^2 = b^2 = (ab)^2 = 1>$.
\end{example}

\begin{thm}
	Пусть $A\subset G$~--- произвольное подмножество группы $G$. Тогда однозначно определена подгруппа $G$, наименьшая по размеру, содержащая $A$ как подмножество.
\end{thm}
\begin{proof}
	Рассмотрим семейство всех подгрупп $G$, содержащих $A$. По результату упражнения \ref{ex:grpint} пересечение всех групп этого семейства будет снова группой. Очевидно, что это наименьшая подгруппа $G$, содержащая $A$.
\end{proof}

\begin{thm}
	Пусть $G$~--- произвольная группа, и $x$~--- её элемент. Тогда подгруппа $<x>$ будет либо свободной группой одного элемента, либо циклической (таким образом произвольная конечная группа всегда содержит в себе циклическую подгруппу).
\end{thm}
\begin{proof}
	Начнём строить последовательность $x, x^2, x^3, \ldots$ Здесь есть два варианта: либо элементы последовательности всегда будут различны, либо встретятся повторяющиеся.
	
	Если найдутся такие числа $k<n$, что $x^k = x^n$, то умножив это равенство с обоих сторон на $x^{-k}$, получим $1 = x^{n-k}$. Это значит, что элементы нашей последовательности будут повторяться циклически:
	\[
	x, x^2, x^3, x^4, \ldots, x^{n-k-1}, 1, x, x^2, x^3, \ldots
	\]
	Произведение любых двух элементов этой последовательности даёт так же элемент этой последовательности:
	\[
	x^a x^b = x^{a + b} = x^{(a+b)\Mod (n-k)}
	\]
	Здесь $\Mod$ обозначает остаток от деления. Обратным элементом для $x^a$ будет являться $x^{a+n-k}$. Таким образом элементы $1, x, x^2, \ldots, x^{n-k-1}$ образуют циклическую группу. Очевидно, что это наименьшая подгруппа $G$, содержащая $x$.
	
	Если же элементы последовательности $x, x^2, \ldots$ все будут различны, то подгруппа $<x>$ будет так же содержать обратные элементы $x^{-1}, x^{-2}, \ldots$ и нейтральный элемент 1. Но это в точности свободная группа $F_{\{x\}}$.
\end{proof}

\begin{definition}
	Отображение групп $h:G\to H$ называется \term{гомоморфизмом}, если $h(ab) = h(a)h(b)$. Если гомоморфизм сюръективен, то он называется \term{мономорфизмом}, если инъективен~--- \term{эпиморфизмом}, если биективен~--- гомоморфизмом. Гомоморфизмы вида $h:G\to G$ (то есть группы в саму себя) называются \term{автоморфизмами}.
\end{definition}

