\section{Сочетания}

<<$k$-сочетание>>~--- это просто другое название для термина <<$k$-элементное подмножество>>, которое по историческим причинам (хотя многие считают, что так удобнее) принято в комбинаторике. Нас будет интересовать количество $k$-сочетаний взятых из множества $[n]$.

\begin{definition}
\term{Биномиальным коэффициентом} $n \choose k$ называется число сочетаний из $[n]$ по $k$.
\end{definition}

Вместо обозначения ${n \choose k}$ в российской и французской литературе исторически чаще используется обозначение $C^k_n$, однако оно кажется мне менее удобным в случаях если вместо величин $n$ и $k$ используются какие-то длинные выражения. По этой причине мы будем придерживаться общемирового обозначения.

Значения ${n \choose 1} = n$ и ${n \choose n} = 1$ очевидны. Так же удобно полагать, что ${n \choose 0} =1$ (пустое множество всего одно, соответственно есть лишь один способ его выбрать). 

\begin{exercise}
На листочке в клетку начерчен прямоугольник со сторонами $m$ и $n$. Мы двигаемся из левого нижнего угла прямоугольника в правый верхний, сдвигаясь за шаг либо на одну клетку вверх, либо на одну клетку вправо. Сколько всего существует таких путей? Сведите задачу к вычислению количества сочетаний, как их считать я расскажу ниже.
\end{exercise}

\begin{exercise}
Будем считать, что в условиях прошлой задачи $m=n$. Сколько существует путей из левого нижнего угла в правый верхний таких, что они не пересекают диагонали квадрата? (Подсказка: возьмите путь, пересекающий диагональ, и отразите его начало до пересечения относительно этой диагонали).
\end{exercise}

\begin{thm}
$$\sum_{k=0}^n{n \choose k} = 2^n$$
\end{thm}
\begin{proof}
В левой части этого выражения строит общее количество всех подмножеств множества $[n]$. В правой части на самом деле написано то же самое: $2^n$ есть мощность булеана, как вы видели в~\S~3.1.
\end{proof}

\begin{thm}
$${n \choose k} = {n \choose n-k}$$
\end{thm}
\begin{proof}
Выбрать $k$ элементов из множества $[n]$ это всё равно что выбрать $n-k$ элементов, которые мы оставим в множестве.
\end{proof}

\begin{thm}
$${n \choose k} = {n-1 \choose k-1} + {n - 1 \choose k}$$
\end{thm}

\begin{proof}
Рассмотрим все $k$-сочетания из множества $[n]$. Сам элемент $n$ может либо принадлежать выбранному подмножеству, либо не принадлежать. В первом случае количество сочетаний будет равно $n-1\choose k$, во втором случае, поскольку один элемент сочетания уже фиксирован, это будет величина $n-1\choose k - 1$.
\end{proof}

Последняя теорема, как и в случае чисел Стирлинга, позволяет вычислять биномиальные коэффициенты. Однако, для сочетаний мы можем записать и явную формулу.

\begin{definition}
\term{$k$-размещением} мы назовём некоторый упорядоченный набор, состоящий из некоторых $k$ элементов множества $[n]$. Количество размещений из $n$ по $k$ будем обозначать как $n^{\lfloor k\rfloor}$.
\end{definition}

\begin{thm}
$$n^{\lfloor k \rfloor} = \frac{n!}{(n-k)!}$$
\end{thm}
\begin{proof}
Доказательство практически дублирует доказательство для количества перестановок. На первую позицию мы можем поставить один из $n$ элементов. На вторую позицию один из оставшихся $n-1$ элементов. На третью~---один из $n-2$ элементов. Однако в отличии от перестановок, теперь нам надо разместить лишь $k$ элементов, а не все $n$, поэтому мы этот процесс оборвём на $k$-том шаге. В итоге получаем выражение
$$n^{\lfloor k \rfloor} = n (n-1)  (n-2) \ldots (n-k+1)$$
Если это выражение умножить и разделить на $(n-k)!$, получим утверждение теоремы.
\end{proof}

Само обозначение $n^{\lfloor k \rfloor}$ на самом деле почти всегда используется просто для обозначения произведения $k$ подряд убывающих чисел. Эту величину часто называют \term{убывающим факториалом}. В полной аналогии вводится и \term{возрастающий факториал}:
$$n^{\lceil k \rceil} = n(n+1)(n+2)\ldots(n+k-1)$$

\begin{exercise}
Покажите, что $n^{\lfloor k \rfloor} = (n-k+1)^{\lceil k \rceil}$
\end{exercise}

\begin{thm}
$${n \choose k} = \frac{n!}{k!(n-k)!}$$
\end{thm}
\begin{proof}
$k$-расстановку мы можем получить, вначале выбрав $k$-элементное подмножество $[n]$, а затем упорядочив его. Всего существует $n\choose k$ способов выбрать такое подмножество. Способов упорядочить выбранное~--- $k!$. Таким образом получаем соотношение
$$n^{\lfloor k \rfloor} = k!{n\choose k}$$
Поделив обе части на $k!$ и подставив выражение для $n^{\lfloor k \rfloor}$, получаем утверждение теоремы.
\end{proof}

\begin{exercise}
Теоремы 3.22 и 3.23 можно доказать, пользуясь явным представлением биномиального коэффициента, полученным в 3.25. Сделайте это.
\end{exercise}

\begin{exercise}
Чему равно $$\sum_{k=m}^n k^{\lfloor m\rfloor} {n\choose k}$$
\end{exercise}

\begin{exercise}
Докажите, что $${n\choose m}{n-m\choose k} = {n\choose k}{n-k\choose m}$$
\end{exercise}

\begin{exercise}
Докажите, что $${n\choose k-1}{n\choose k+1} \le {n\choose k}{n\choose k}$$
\end{exercise}

\begin{thm}
$$(x+y)^n = \sum_{k=0}^n {n \choose k} x^{n-k} y^k$$
\end{thm}
\begin{proof}
Давайте вначале для наглядности распишем степень подробно:
$$(x+y)(x+y)\ldots(x+y)$$
Раскроем все скобки (если пока не понятно как, то потренируйтесь на каких-то частных случаях типа $n=3$). Раскрытие скобок можно интерпретировать так, что из каждой скобки мы выбираем либо $x$, либо $y$. Все слагаемые в полученной сумме будут иметь вид $x^iy^j$, $i+j=n$ с каким-то коэффициентом, появляющимся за счёт того, что некоторые слагаемые вида $x^iy^j$ появляются несколько раз.

Слагаемое $x^n$ появится лишь один раз в случае, если из всех скобок мы выберем $x$. Слагаемое $x^{n-1}y$ появляется, если мы выбираем $x$ из всех скобок, кроме одной. Эту одну скобку, из которой мы выбираем $y$, мы можем выбрать одним из $n$ способов. По аналогии $x^{n-2}y^2$ появится $n\choose 2$ раз, поскольку мы выбираем уже две скобки, из которых мы возьмём $y$. Продолжая по аналогии приходим к утверждению теоремы.
\end{proof}

\begin{exercise}
Приведённую теорему можно доказать по индукции. Будет полезным проделать это самостоятельно.
\end{exercise}

Приведённая теорема даёт нам ещё один способ подсчитать количество подмножеств множества $[n]$:

$$\sum_{k=0}^n {n \choose k} = \sum_{k=0}^n {n \choose k}1^{n-k}1^k = (1+1)^n = 2^n$$

\begin{exercise}
Докажите равенство
$$3^n = \sum_{k=0}^{n} 2^k {n \choose k}$$
\end{exercise}

Последняя задача в свете теоремы 3.26 решается элементарно, однако я замечу, что такие задачи часто даются на олимпиадах для тех школьников, которые ещё по возрасту не могут знать таких теорем. Предполагается, что школьники могут доказать утверждение по индукции (и некоторые действительно могут). Это одна из причин по которой автор ненавидит олимпиадные задачки и <<задачки на логику>>, которые часто дают на собеседованиях~--- большинство таких задач изначально появляются как тривиальные следствия каких-то более общих результатов, а потом уже задним числом оказывается, что задачу в общем-то можно было бы решить и пользуясь лишь школьной математикой. Однако если сравнивать пользу от решения такой задачи по индукции с пользой от изучения общей теоремы, то последнее явно выигрывает, хотя и не помогает на собеседованиях и олимпиадах.

Сочетания часто используются так же вот в каком ключе. Предположим, мама нам сказала: <<Пойди в магазин и купи $n$ каких-нибудь пирожков>>. Мы приходим в магазин, а там продаётся $k$ наименований пирожков. Сколько всего способов у нас есть удовлетворить мамин запрос? Задача в такой постановке приводит нас к понятию \term{сочетаний с повторениями}, поскольку искомые наборы могут содержать повторяющиеся элементы.

Для решения задачи предположим, что пирожки разного вида мы разложили по разным пакетам (я видел, что в ларьках у метро именно так часто и делают). Схематически мы будем разделять пакеты вертикальной чертой $|$, а пирожки (или, более общо, элементы множества), кружочками $\circ$. Для примера давайте считать, что у нас всего имеется $k=5$ видов пирожков, а купили мы $n=6$ пирожков, причём из них было два пирожка первого вида, три третьего и один четвёртого. Остальных пирожков мы не покупали. На схеме это будет выглядеть как
$$|\circ\circ||\circ\circ\circ|\circ||$$
Теперь мы можем догадаться, как подсчитать общее количество различных сочетаний с повторениями: достаточно просто подсчитать общее количество возможных схем такого вида. В схеме, приведённой выше, если отбросить крайние чёрточки $|$, получится $n+k-1$ различных позиций, на которых могут стоять чёрточки либо кружки. Причём мы точно знаем, что кружков всего будет $n$ штук, а чёрточек $k-1$. Чтобы получить какую-то конкретную схему, нам надо выбрать $n$ позиций под кружки, а остальные позиции мы занимаем чёрточками. Итого для количества сочетаний с повторениями мы имеем выражение
$${n+k-1 \choose n} = {n+k-1\choose k - 1}$$

\begin{exercise}
Докажите, что существует $2^{n-1}$ способов представить число $n$ в виде суммы ненулевых слагаемых. Задача довольно сложная, поэтому дам некоторые наводки. Вначале следует разобрать случай, когда имеется ровно $k$ слагаемых. Подход здесь может быть аналогичным задаче с булочками выше, но надо учитывать то, что чтобы слагаемые были ненулевыми, мы не можем поставить подряд две черты, не поставив между ними кружок. Способов сделать это $n-1\choose k -1$ (докажите!). Отсюда уже довольно легко выводится и результат первоначальной задачи.
\end{exercise}

