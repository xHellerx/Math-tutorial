\section{О честных выборах}

Отвлечёмся немного от теории и обратимся к теме, которая на первый взгляд к серьёзной математике почти не имеет отношения: к политическим выборам. Сделаем мы это с одной стороны ради чистого интереса и веселья, а с другой стороны для того, чтобы на простом интуитивном примере ввести понятие \term{ультрафильтров}, которые впослеедствии постужит нам добрую службу.

Предположим, что у нас имеется кандидат в президенты, которого готовы отдать голоса 26\% избирателей. Вслед за ним идёт кандидат с 12\%, за ним с 11\% и дальше идёт большое количетво кандидатов, за которые готовы голосовать менее 10\% населения. Предположим теперь, что самый популярный кандидат~--- фашист, людоед и сволочь, и в общем-то кроме 26\% его сторонников никто больше за него голосовать не стал бы и вообще в страшном сне видит его победу. Одна проблема: его противники разбиты на разные идеологические лагери и не имеют единого кандидата. Может быть большинство голосует даже не за своего кандидата, а "лишь бы против людоеда". Вот только против они голосуют неорганизованно. По результатам выборов простым большинством людоед побеждает, хотя это совершенно не отражает волю большинства.

Чтобы таких исторических трагедий не происходило, была придумана двухступенчатая система голосования. В первом этапе голосования определяются два фаворита, а во вотором туре голосовать можно уже только за одного из лидеров. Назовём условно кандидата с 12\% голосов <<либералом>>. Не смотря на то, что в первом туре он значительно уступает людоеду, они оба проходят во второй тур, и уже во втором туре все противники канибаллизма голосуют за либерала. Не потому что он очень им нравится, а потому что они не хотят допустить фашиста во власть. Итог: либерал побеждает с 74\% голосов против всё тех же 26\% у фашиста.

Вроде бы нас двухступенчатая система спасла от трагедии. Но так ли хороша она на самом деле? Фашист ведь не дурак, и при наличии поддержки его значительной долей избирателей, он может ввести на выборы своего искусственного оппонента, а в действительности единомышленника. Если он так поступит, то, скажем, половина избираталей уйдёт к <<аппоненту>> и оба получат примерно по 13\% голосов. Это всё равно больше, чем у либерала, и в итоге во второй тур проходят два практически идентичных людоеда, один из которых становится президентом, а второй премьер-министром.

Такое большое преимущество конечно радко случается, хотя бывает. Предположим, что за фашиста отдают голоса всего 14\% населения. В этом случае ему победить будет уже сложнее, но он всё равно имеет возможность манипулировать выборами. Фашист может создать искуственного конкурента своему наиболее опасному аппоненту, введя на выборы ещё одного либерала. Пусть этот второй либерал не будет популярен, но даже если он отъест хотя бы 2\% от первого либерала, во второй тур либералы уже не пройдут. А кандидат, следующий за либералом с 11\% голосами, вероятно, очень слаб. Например, он сталинист или ЛГБТ-активист, а в России многие предпочтут голосовать скорее за фашиста, нежели за гомосексуалиста (все же нормальные пацаны). Опять же фашист побеждает.

Помимо того, что кандидаты в президенты могут манипулировать выборами, вводя своих фиктивных кандидатов, сами избиратели так же действуют часто тактически. При двуступенчатой системе голосования у действительно симпатичного кандидата редко есть хоть какие-либо шансы на успех и поэтому голосовать за него нет смысла. Куда полезнее определиться с тем, чью победу допускать не хотелось бы вообще никак и бороться против него. В этом случае избирателю разумно определить список хоть как-то приемлемых кандидатов, оценить того, у которого наивысшие шансы на проход во второй тур, и голосовать за него. Многие исследования подтверждают, что в большинстве случае именно так и происходит: люди голосуют не за своего кандидата, а за самого серьезного оппонента тому, кто им неприятен.

Наиболее подвержденны такитческим голосованиям выбора в парламен, который как правило формируется пропорционально: какая доля населения отдала свои голоса за партию, такую долю партия и получит в парламенте. При этом доля может быть не произвольной, а существует некий <<проходной барьер>>. В России, например, этот барьер составляет 5\%, и если партия не получает 5\% голосов, то она вообще не получает представительства в Думе. Необходимость в каком-либо барьере существует чисто техническая: если в парламенте имеется всего 100 мест, а партия набрала 5 голосов из двух миллионов, то очевидно, что места ей не хватит. Другое дело, что правительство часто завышает барьер куда выше, нежели того требуют технические соображения. Цель, которая при этом преследуется, официально состоит в том, чтобы избежать дробления парламента на маленькие фракции и тем самым позволить парламенту более эффективно работать, а так же в том, чтобы не допустить в парламент радикалов, маргиналов и экстремистов.

На практике часто получается так, что проходной барьер оказывается средством манипулирования выборами. На выборах в ГосДуму в 1995 году 45\% голосов набрали те партии, которые не смогли предолеть пятипроцентного барера. Таким образом мнение половины населения при распределении мест в Думе было вообще никак не учтено и парламентские места по сути были отданы людям, против которых голосовала половина населения. На выборах 2011 года проходной барьер был повышен до 7\%, что по сути перекрывало путь в правительство любым оппозиционным партиям.

Процентный барьер могут использовать не только политические партии в своих целях, он так же предполагает широкие возможности по тактическому голосованию для избирателей. Предположим, что на 100 мест парламента претендует четыре партии~--- две партии имеют по 44\% голосов, одна оппозиционная партия 8\% голосов и одна оппозиционная 4\%. При проходном барьере в 5% последняя партия не проходит в парламент. После голосования без учёта проигравшей партии, две крупных партии получат по 46 мест каждая, и оппозиционная партия, прошедшая барьер получит 8 мест.

Предположим теперь, что часть оппозиционеров (скажем, 2\%) решила отдать свои голоса менее популярной партии. В этом случае обе партии получают по 6\% и обе проходят в парламент. На каждую партию получается меньше мест (6 вместо восьми), однако в сумме оппозиция представлена уже 12-ю парламентариями, а не 8-ю. Крупным партиям так же приходится немного стесниться: вместо 92 мест на двоих они теперь имеют только 88 мест.

Понятно, что когда политики пытаются манипулировать выборами~--- это плохо. Но на самом деле так же плохо и когда избиратели манипулируют выборами, так в этом случае выборы превращаются из процесса определения мнения населения в интеллектуальную стратегическую игру где побеждает не тот, кто дейстивительно предпочтителен обществом, а тот, кто переиграл оппонента. Это явно не соответствует заявленной цели демократических выборов.

Проблема тут кроется не в избирателях или политиках, которые манипулируют выборами, а в самих правилах игры. Как мы видели, двухступенчатые выборы могут защитить нас от ситуации, при которой побеждает наименее желательный кандидат. Можно придумать и другие системы выборов, которые ещё более надёжды.

Простейшая система голосований~--- <<одобрительные выборы>>, когда избиратель указывает в биллютене не лучшего по его мнению кандидата, а ставит галку напротив всех тех кандидатов, которые в принципе ему приемлемы. Кандидат, который примлем по мнению большинства избирателей, побеждает. При такой системе выборов введение любых новых кандидатов на выборы никак не может повлиять на выбор победителя, поскольку учитывается не доля отданных голосов, а общее количество одобрений.

В то же время такая система выборов оказывается значительно хуже, чем двуступенчатое голосование. Чтобы увидеть это, давайте представим, что у нас есть четыре кандидата: $A$, $B$, $C$ и $D$ и три избирателя. Предпочтения каждого избирателя определяются некоторой перестановкой множества кандидатов.

Пусть первый избиратель имеет предпочтения $A>B>D>C$, второй $B>C>D>A$ и третий $C>A>D>B$. При таких предпочтениях кандидат $D$ оказывается самым худшим: при выборах лишь между двумя кандидатами, он всегда проиграет. В то же время если каждый из избирателей обозначит всех кандидатов как приемлемых, кроме своего наименее желательного кандидата, то кандидат $D$ наберёт наибольшее количество голосов и победит.

Одобрительные выборы имеют ещё и такую проблему: может случиться такое, что кандидат, однозначно предпочитаемый более чем половиной избирателей, в итоге не победит на выборах. Например, пусть 55 избирателей имеют предпочтения $A>B>C$, 35 избирателей $B>C>A$ и 10 избирателей $C>B>A$. 35 избирателей, однозначно предпочитающих кандидата $A$~--- это больше половины. Однако, если все кандидаты считают приемлемыми двух кандидатов из трёх, то в итоге кандидат $A$ окажется приемлемым 45 раз, кандидат $B$ 100 раз и $C$ 45 раз. Побеждает $B$.

К этому явлению можно относиться по-разному. Кто-то скажет, что голосование с таким свойством неприемлемо и при нём побеждают только очень средние кандидаты. Кто-то напротив, скажет, что это вид голосования, максимально учитывающее итересы каждого избирателя, выбирая пусть не лучшего кандидата, но приемлемого.

\begin{exercise}
Можно придумать и систему голосования, при которой каждый избиратель не просто сообщает приемлем ли ему кандидат, но выставляет каждому кандидату оценку как в школе. Все оценки в итоге суммируются и побеждает кандидат, получивший набольшую оценку. Покажите, что при таком голосовании опять же может победить самый слабый кандидат (в том смысле, что при выборах одного из двух он проиграл бы каждому другому кандидату), а так же что кандидат, которого однозначно предпочитает больше половины избирателей, может не победить.
\end{exercise}

Голосование с выставлением оценок можно интерпретировать так, что каждый избиратель оценивает полезность данного кандидата лично для себя. Побеждает кандидат, суммарная полезность которого наиболее высока. Можно провести аналогию, что оценка кандидата~--- это то, сколько условно будет зарабатывать избиратель при победе данного кандидата. После голосования побеждает кандидат, суммурный доход населения при котором будет максимальным. Может показаться, что это оптимальная в экономическом смысле система голосования, однако это не так: никто не гарантирует, что большие доходы будут распределены справедливо. Вполне может быть, что при одном кандидате 100 человек зарабатывает по 50 рублей (условно) и один 200 рублей, а при другом 100 человек зарабатывают по рублю, а один целлый миллион. Победит конечно последний кандидат, но предпочтительнее для большинства, очевидно, первый.

Так же такая методика очень подверждена тактическому голосованию, поскольку она позволяет несправедливо занижать баллы кандидатам.

\begin{exercise}
Если ограничить голосование с оценками так, что каждый избиратель должен упорядочить кандидатов (то есть поставить им всем разные оценки от 1 до $m$, где $m$~--- это число кандидатов), то получится метод голосования, называемый методом Борда. Докажите, что при этом подходе слабейший кандидат никак не может победить. (Подсказка: используйте двойной счёт). Все остальные недостатки этот подход, однако, сохраняет.
\end{exercise}

