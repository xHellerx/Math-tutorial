\section{Формализм}

До сих пор все наши рассуждения были главным образом интуитивными, мы аппелировали к каким-то физическим образам и вводили нестрогие вспомогательные понятия вроде множеств. Этот подход нельзя назвать совсем строгим с математической точки зрения. В этой главе мы закроем эту дырку и введём все понятия, которые уже были нами введены, формально.

Состоять наше изложение логики будет из трёх частей: языка (как мы записываем предложения), синтаксиса (правила вывода одного предложения из другого) и семантики (наделение предложений предполагаемым смыслом).

Как мы уже отмечали, когда мы даём какое-то определение, мы всегда вынуждены пользоваться другими определениями. В конечном итоге мы либо обязаны ввести какое-то понятие, которое мы никак не определяем. Это называется <<принципом Мюнгхаузена>>: если бы мы не начинали построение математики от какого-то неопределяемого понятия, то получилось бы что наши определения были как-то зависимы друг от друга: условно говоря определение А базировалось бы на определении Б, а определение Б на определении А, и это в самом явном случае (зависимости могли бы быть самыми сложными теоретическими, но такие определения всегда были бы ошибочны).

В качестве понятия, которое мы никак не будем определять, у нас будет выступать <<символ>>. С точки зрения интуиции, символ~--- это некоторая закорючка на бумаги. С точки же зрения логики это понятие, которое просто мы принимаем без попыток понять что это.

\begin{definition}
\term{Алфавитом} назовём набор символов.
\end{definition}

\begin{example}
Примером алфавитов может служить русский или английский алфавит.
\end{example}

\begin{example}
Для нужд логики мы определим алфавит, состоящий из символов $\land$, $\lor$, $\to$, $\neg$, $\oplus$, $\leftrightarrow$, а так же всех символов английского языка, как строчных, так и заглавных. Сам этот алфавит будем обозначать как $\Sigma$.
\end{example}

\begin{definition}
\term{Строкой} (или так же \term{словом}) называется конечная упорядоченная последовательность сиволов (возможно, пустая) некоторого алвавита. Пустая строка для удобства обозначается как $\epsilon$.
\end{definition}

Строками алвавита $\Sigma$, определённого выше, являются например такие выражения как $\land\land P\oplus\leftrightarrow$. Строками русского алфавита будут такие последовательности как <<аплотфдц>>, а английского такие как <<shehs>>. Это совершенно бессмысленный набор символов, и отсюда ясно, что нам необходимо как-то из всех возможных строк выделить допустимые.

\begin{definition}
Набор слов (возможно, бесконечный) называется \term{языком}.
\end{definition}

Простейший, но одновременно с тем и почти бесполезный способ задания языка~--- это простое перечисление всех строк алфавита. В каких-то частных случаях это было бы возможно. Например, мы могли бы сконструировать язык, описывающий все возможные с положения игры крестики-нолики. В качестве алфавита мы выбрали бы набор $\{x, o, ?,1, 2\}$, а в качестве языка условились бы называть все строки длины 10 этого алфавита, в которых первым символом идёт либо 1 либо 2 (что обозначает игрока, которому принадлежит ход), а оставшиеся символы обозначали бы подряд все клетки поля, где помимо крестиков и ноликов мы могли бы ставить символ $?$ для незанятых клеток. Примером такого слова может служить строка 20?0?x???xx~--- она описывает ситуацию, изображенную в таблице~1.11. Хотя допустимых строк в таком языке и довольно много, их очевидно все можно перечислить.

\begin{table}[h]
\centering
\begin{tabular}{c | c | c}
o & & o\\
\hline
  & x & \\
\hline
 & x & x
\end{tabular}
\caption{Ход второго игрока}
\end{table}

Подход с перечислением всех возможных строк, очевидно, не сработает в общем случае. Перечислить все возможные предложения логики без использования каких-то специальных механизмов явно невозможно, не говоря уже о русском или английском языках.

\begin{exercise}
Задайте алфавит
\end{exercise}

\begin{exercise}
\term{Грамматикой} называется некоторое формальное описание структуры допустимых слов языка.
\end{exercise}

Это довольно нечёткое определение. Когда мы выше сказали, что язык крестиков-ноликов~--- это строки длины десять, заполненные определёнными элементами, мы по сути задали грамматику языка.