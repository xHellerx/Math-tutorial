\section{Вычислительный аспект}

В первом параграфе мы определили арифметические операции над натуральными числами, но однако ни сказали ни слова о том, как реально вычислять результат от их применения. Можно, конечно, использовать аксиомы напрямую. Так, для сложения $m+n$ в соответствии с акмиомами Пеано нам потребуется $n$ раз прибавить 1 к числу $m$. Скажем, вот так может начинаться процесс сложения $123+456$:
$$123+456 = 124 + 455 = 125 + 454 = 126 + 453 = \ldots$$
Очевидно, что этот способ никуда не годится. В этом параграфе мы рассмотрим каким образом можно проводить арифметические вычисления более-менее эффективно, но прежде введем новую для нас удобную нотацию.

Пусть $\{x_i\}$~--- некоторая последовательность и мы хотим сложить все элементы этой последовательности подряд начиная $x_a$ и заканчивая $x_b$. Кратко мы будем записывать эту сумму с помощью символа $\sum$:
$$\sum_{i=a}^b x_i = x_a + x_{a+1} + \ldots + x_b$$
Аналогичную краткую запись мы введем и для произведения:
$$\prod_{i=a}^b x_i = x_a \cdot x_{a+1} \cdot \ldots \cdot x_b$$
Для пронумерованного семейства множеств $\{S_i\}$ аналогично введем краткое обозначение для объединения и пересечения:
$$\bigcup_{i=a}^b A_i = A_a \cup A_{a+1}\cup\ldots\cup A_b$$
$$\bigcap_{i=a}^b A_i = A_a \cap A_{a+1}\cap\ldots\cap A_b$$
В полной аналогии можно обозначать конъюнкцию, дизъюнкцию и исключающее или  для логических высказываний да и вообще многие другие операции, но мы не будем на этом лишний раз останавливаться~--- эти обозначения и так очевидны и понятны.

Пусть мы теперь хотим проссумировать не все элементы $x_i$ в каком-то диапазоне, а в точности те значения $x_i$, где $i$ принадлежит некоторому наперед заданному множеству $S$. Это множно кратко записать так:
$$\sum_{i\in S} x_i$$
Аналогично можно записывать и прочие операции.

Пользуясь введенной нотацией любое натуральное число $n$, для представления которого требуется $k$ разрядов, можно записать в системе счисления с основанием $b$ таким образом:
$$n = \sum_{i=0}^{k-1} r_i b^i = r_{k - 1} b^{k-1} + \ldots + r_1 b + r_0$$
Или даже, если посчитать, что при $i\ge k$ все $r_i = 0$, можно избавиться в этой сумме от величины $k$:
$$n = \sum_{i=0}^\infty r_i b^i = \sum_{i\in\mathbb{N}} r_i b^i$$
Символом $\infty$ мы абстрактно обозначили <<бесконечность>>, что означает, что мы будем суммировать по всем значениям $i$ вообще. В общем случае вохможность сложения бесконечного числа чисел вызывает сразу ряд вопросов, однако в нашем случае мы знаем, что лишь конечное число слагаемых $r_ib^i$ будет отлично от нуля, так что реально здесь суммируется по сути конечное число значений и проблемы с этим не возникает.

Пусть мы хотим сложить числа $a$ и $b$. Их разряды в системе счисления с основанием $d$ мы обозначим как $a_i$ и $b_i$ соответственно. Тогда, если вспомнить, что сложение может осуществляться в любом порядке и мы можем как угодно переставлять в суммах скобки (см.~\S3.1), мы элементарно получаем следующее соотношение:
$$\left(\sum_{i=0}^\infty a_i d^i\right) + \left(\sum_{i=0}^\infty b_i d^i\right) = \sum_{i=0}^\infty (a_i + b_i) d^i$$
Буквально здесь говорится, что мы можем складывать числа поразрядно. То есть для нашего примера с 123 и 456 мы можем отдельно сложить 1+4, 2+5 и 3+6. Результатом будет 579 (если вам непонятны эти рассуждения, распишите эти числа как сумму в десятичной системе счисления и проведите вычисления аккуратно).

Если мы теперь попытаемся сложить 579 и 123, то у нас выйдет проблема: $3+9=12$, что больше, чем основание системы счисления. Это значит, что 12 надо представить как $1\cdot10 + 2$, и теперь единица переходит в старший разряд. В итоге вместо $7+2$ мы должны во второй разряд записать $7+2+1=10$, что опять не умещается в систему счиелния. Снова единица переёдет уже в третий разрад: $5+1+1$. Результат: 702.

Мы не будем останавливаться на этом подробно, так как все эти вещи соверешенно тривиальны. В школах используется именно этот способ сложения, только для удобства записи числа записываются в столбик, разряд под разрядом, только там редко объясняют почему такой способ сложения вообще работает. Мы то только что доказали.

Вычитание практически аналогично сложению и мы не будем его рассматривать отдельно.

Для произведения формула получается интереснее (мы могли бы получить много разных вариантов, но именно этот вариант простой перестановки скобок~--- стандартный школьный):
$$\left(\sum_{i=0}^\infty a_i d^i \right)\left( \sum_{j=0}^\infty b_j d^j \right) = \sum_{i=0}^\infty \left(\sum_{j=0}^\infty a_i b_j d^j \right)d^i$$
Это выражение в точности дублирует умножение в столбик~--- каждый разряд числа $a$ по отдельности умножается на число $b$ целиком (опять же поразрядно). Не будем вдаваться в скучные подробности умножения в столбик, а обратим внимание на следующий важный аспект:

\begin{exercise}
Пусть даны два $n$-разрядных числа $a$ и $b$. Для их умножения приведенным способом потребуется $n^2$ операций умножения отдельных разрядов.
\end{exercise}

Если взять числа 23 и 45 и сложить их, то нам потребуется сложить разряды $2+4$ и $3+5$. Для умножения же этих чисел, потребуется вычислить произведения $2\cdot 4$, $2\cdot 5$, $3\cdot 4$ и $3\cdot 5$. Если бы мы взяли не двузначные числа, а трезначные, то при сложении нам потребовалось сложить три разряда, а при умножении перемножить девять разрядов. Для сложения 100-разрядных чисел потребуется 100 операций сложения разрядов, а для умножения их же~--- 10000 операций умножения.

Как видно, сложность умножения (если выражать сложность в количестве операций)  значительно выше, нежели сложность сложения, причем чем больше числа по разрядности, тем значительнее возрастает сложность умножения, в отличие от сложения.

