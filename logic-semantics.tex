\section{Семантика}

{\bfseries Определение.} {\slshape Моделью} называется множество высказываний.

Если в модели $M$ истинно высказывание $s$, то это обозначается как $M\models s$.

Если любая формула теории $T$ истинна в модели $M$, то говорят, что $M$ является моделью $T$.

У каждой теории может быть много моделей, которые могут иметь как вообще различный набор истинных высказываний, так и просто разную трактовку. Рассмотрим простенькую теорию, для которой $T\vdash \neg a \vee (b \wedge c)$ и $T\vdash a \oplus d$.

Можно рассмотреть модель этой теории, в которой $a$ = «светит солнце», $b$ = «девки по улице гуляют», $c$ = «птички поют», $d$ = «Вася по улице гуляет». Формулы теории $T$ соответственно обозначают, что если на лице светит солнце, то девки гуляют и птички поют, и что у Васи аллергия на Солнце, и пока все гуляют, он дома сидит и книжку по математике читает. При этом в нашей модели $M_0\models a, b, c, \neg d$. Это один из вариантов, но важно, что для каждого высказывания мы точно можем определить его истинность в данной модели.

Или можно рассмотреть модель, в которой $a$ = «растет курс рубля», $b$ = «повышается импорт», $c$ = «снижается спрос на отечественную продукцию», $d$ = «растет производство внутри страны». Теоремы на этот раз можно интерпретировать как то, что из роста курса рубля повышается импорт и снижается спрос на отечественную продукцию, а так же то что рост курса рубля и рост производства внутри страны не могут происходить одновременно. Это очень утрированная теория с точки зрения экономики, но для учебника математики пойдет. Можно считать, что в нашей конкретной модели $M_1\models \neg a, b, \neg c, d$.

Как видно, модели одной и той же теории могут различаться не только физической интерпретацией, но и истинностью отдельных высказываний. В примере выше у нас в одной модели было $M_0\models a$, а в другой $M_1\models \neg a$, и обе эти модели удовлетворяли теории $T$. Отсюда, например, можно сразу заключить, что высказывание $a$ не выводимо само по себе в теории $T$, что обозначается как $T\not\vdash a$.

Множество всех моделей теории $T$ обозначается как $\mathrm{Mod}(T)$. Например, для рассмотренной выше теории,\\
$\mathrm{Mod}(T) = \{\{a, b, c, \neg d\}, \{\neg a, b, c, d\}, \{\neg a, \neg b, c, d\}, \{\neg a, b, \neg c, d\},\\ \{\neg a, \neg b,\neg c, d\}\}$, плюс у каждой модели могут быть разные интерпретации.

{\bfseries Упражнение.} Докажите, что для противоречивой теории не существует модели.

Возможна и обратная задача. Если задано множество моделей, то по ним можно построить теорию, которая в точности будет удовлетворять этим моделями и никаким другим. Если $S$ — множество моделей, то теория, по ним построенная обозначается как $\mathrm{Th}(S)$.

{\bfseries Упражнение.} Пусть $S = \{\{a, \neg b, c\}, \{a, \neg b, \neg c\}, \{\neg a, b, \neg c\}\}$. Аксиоматизируйте $\mathrm{Th}(S)$ несколькими разными способами.

{\bfseries Теорема.} Если $T_1 \subset T_0$, то $\mathrm{Mod}(T_0) \subset \mathrm{Mod}(T_1)$.

{\bfseries Доказательство.} Каждую формулу теории можно считать предикатом, который задает подмножество на множестве моделей. Запись $T_1 \subset T_0$ можно трактовать как то, что $T_0$ получается из $T_1$ добавлением предикатов, которые задают подмножество в множестве $\mathrm{Mod}(T_1)$. \qed

Добавляя к теориям новые формулы и наблюдая, как это влияет на множество моделей самой теории, можно делать выводы о самой теории. Например, если $\mathrm{Mod}(T, \phi) = \mathrm{Mod}(T, \psi)$, то есть при добавлении к теории формул $\phi$ и $\psi$ мы получаем одни и те же множества моделей, то можно говорить в некотором смысле об эквивалентности формул $\phi$ и $\psi$ — в любой модели теории они будут либо одновременно истинны, либо одновременно ложны.

{\bfseries Теорема.} Если $T\vdash \phi \leftrightarrow \psi$, то $\mathrm{Mod}(T, \phi) = \mathrm{Mod}(T, \psi)$.

{\bfseries Доказательство.} По правилу сокращения эквиваленции $T, \phi\vdash \psi$ и $T, \psi \vdash \phi$, то есть при добавлении к теории $T$ формул $\phi$ либо $\psi$, мы все равно в результате получаем одни и те же множества формул нашей теории. \qed

А вот обратное вообще говоря не всегда верно. Если две формулы одновременно либо ложны либо истинны (множества моделей для них совпадают), то еще совершенно не факт, что это возможно вывести в нашей теории по правилам вывода. По этой причине различают два понятия — {\slshape синтаксис}, имеющий дело с выводимостью, и {\slshape семантику}, имеющую дело с моделями. В случае эквивалентности, синтаксическая эквивалетность утверждает, что формулы могут быть выведены одна из другой и наоборот, а семантическая эквивалентность утверждает, что они всегда либо одновременно ложны, либо одновременно истинны.

По аналогии говорят, что $\psi$ является {\slshape семантическим следствием} $\phi$, если всегда, когда истинно $\phi$, истинно так же и $\psi$, что в терминах моделей записывается как $\mathrm{Mod}(T, \phi) \subset \mathrm{Mod}(T, \psi)$.

Опять же, если $\psi$ выводимо из $\phi$ в теории $T$, то оно же автоматически является и синтаксическим следствием $\phi$. Обратное не всегда верно.

{\bfseries Определение.} Теория называется {\slshape полной}, если из семантического следствия вытекает выводимость.

Пример неполной теории мы пока не сумеем на самом деле привести. {\slshape Теорема Гёделя о полноте} утверждает, что любая теория в рамкой той логики, которую мы до сих пор видели, является полной. Теорему эту мы доказывать не будем (вы можете попробовать это самостоятельно), так как в контексте нашего курса она не особо интересна — все последующее изложение у нас будет вестись в контексте неполной теории.

Вернемся к нашим следствиям. Мы сформулировали понятие семантической эквивалентности для моделей, и оказалось, что это тесно связано с обычной операцией эквиваленции. Введем теперь по аналогии с эквиваленцией логическую операцию для следствия, которая бы находилась в соответствии с семантикой нашей теории. Научно ее называют «импликацией», от латинского «imlicatio», что значит «следствие».

{\bfseries Определение.} {\slshape Импликацией} из $\phi$ в $\psi$ (обозначение $\phi\rightarrow \psi$) мы будем называть такую логическую операцию, для которой истинность $\phi\rightarrow\psi$ эквивалентна семантическому следствию $\mathrm{Mod}(T, \phi) \subset \mathrm{Mod}(T, \psi)$.

То есть можно написать так: $\phi\rightarrow\psi = \mathrm{Mod}(T, \phi) \subset \mathrm{Mod}(T, \psi)$

По этому определению для импликации легко построить таблицу истинности. Пусть, например, у нас задана модель $M\models\phi, \neg\psi$. Очевидно, что при этом свойство $\mathrm{Mod}(T, \phi) \subset \mathrm{Mod}(T, \psi)$ не выполняется, и $\phi\rightarrow\psi = 0$. Аналогично можно рассмотреть другие случаи, и получить следующую таблицу истинности:

$\begin{array}{cc|c}\phi&\psi&\phi\rightarrow\psi \\ \hline 0&0&1 \\ 0&1&1 \\ 1&0&0 \\ 1&1&1\end{array}$

Из всего сказанного и таблицы истинности можно тут же выразить следующие очевидные и логичные свойства:

1) $a \rightarrow b = b \vee \neg a$

2) $\neg(a \rightarrow b) = a \wedge \neg b$

3) $a \rightarrow a$

4) $a \leftrightarrow b = (a \rightarrow b) \wedge (b\rightarrow a)$

5) {\slshape Транзитивность:} $((a \rightarrow b) \wedge (b \rightarrow c)) \rightarrow (a \rightarrow c)$

6) $(a \vee b) \wedge (\neg a \vee c) \rightarrow b \vee c$

7) $(a \rightarrow b \wedge c) \rightarrow (a \rightarrow b)$

8) $a \rightarrow b = \neg b \rightarrow \neg a$

А так же правила вывода:

1) Теорема дедукции:

а) $\begin{array}{l}T, \phi\vdash\psi\\ \hline T\vdash\phi\rightarrow\psi\end{array}$

б) $\begin{array}{l}T\vdash\phi\rightarrow\psi\\ \hline T, \phi\vdash\psi\end{array}$

2) Modus ponens:

$\begin{array}{l}\phi\\ \phi\rightarrow\psi\\ \hline \psi\end{array}$

3) Modus tollens:

$\begin{array}{l}\neg\psi\\ \phi\rightarrow\psi\\ \hline \neg\phi\end{array}$

4) Анализ частных:

$\begin{array}{l}\phi\vee\psi\\ \phi\rightarrow\chi\\ \psi\rightarrow\chi\\ \hline\chi\end{array}$

5) Введение эквиваленции:

$\begin{array}{l}\phi\rightarrow\psi\\\psi\rightarrow\phi\\ \hline\phi\leftrightarrow\psi\end{array}$

Ну и свойства, связанные с кванторами:

1) $\forall x, (P\rightarrow Q(x)) = P\rightarrow (\forall x, Q(x))$

2) $\forall x, (P(x)\rightarrow Q) = (\exists x, P(x))\rightarrow Q$

Докажем для примера последнее свойство:

1) $T\vdash\forall x, (P(x)\rightarrow Q)$ — начальное условие

2) $T\vdash P(a)\rightarrow Q$ — правило UI

3) $T, (\exists x, P(x))\vdash P(a)$ — правило EI, примененное к $\exists x, P(x)$

4) $T, (\exists x, P(x)) \vdash Q$ — правило modus ponens для 2) и 3)

5) $T\vdash(\exists x, P(x))\rightarrow Q$ — дедукция

Что и требовалось. И теперь доказательство в другую сторону:

1) $T\vdash(\exists x, P(x))\rightarrow Q$ — начальное условие

2) $T, P(a) \vdash \exists x, P(x)$ — правило EG для P(a)

3) $T, P(a) \vdash Q$ — modus ponens для 1) и 2)

4) $T\vdash P(a)\rightarrow Q$ — дедукция

5) $T\vdash \forall x, (P(x)\rightarrow Q)$ — правило UG

Можно было, впрочем, решить задачу и в лоб:

$\forall x, (P(x)\rightarrow Q) = \forall x, (\neg P(x) \vee Q) = (\forall x, \neg P(x)) \vee Q = (\neg\exists x, P(x))\vee Q = (\exists x, P(x))\rightarrow Q$

{\bfseries Упражнение.} Докажите все остальные правила для импликации, приведенные выше.

{\bfseries Упражнение.} Выясните, в каких случаях неравенства в последнем упражнении §1.4, можно заменить на импликацию.

{\bfseries Упражнение.} Пусть $T\vdash\phi\rightarrow\psi$. Мы каким-то образом работаем с теорией $T$ в предположении $\phi$. Вдруг мы обратили внимание на нашу импликацию, и решили, что работать с предположением $\psi$ вместо $\phi$ было бы удобнее. Мы так и поступаем: забываем о предположении истинности $\phi$ и считаем истинным $\psi$. Покажите, что множество выводимых формул при этом возможно (но не обязательно), сократится.

Последнее упражнение — это пример ошибочных рассуждений и нравственное наставление: так как описано в упражнении никогда нельзя поступать, это в крайней степени порочно и греховно. Чтобы ничего не потерять в теории, мы не должны выкидывать никаких формул и предположений, лишь добавлять гипотезы к уже имеющемуся у нас набору по необходимости, либо использовать теорему о дедукции и получать дополнительные правила импликации.
